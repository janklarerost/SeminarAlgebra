%
% themen.tex -- slide template
%
% (c) 2021 Prof Dr Andreas Müller, OST Ostschweizer Fachhochschule
%
\bgroup
\begin{frame}[t]
\setlength{\abovedisplayskip}{5pt}
\setlength{\belowdisplayskip}{5pt}
\frametitle{Themen}
\vspace{-20pt}
\begin{columns}[t,onlytextwidth]
\only<1-21>{
\begin{column}{0.48\textwidth}
\begin{enumerate}
\item<2-> Geradlinige Bewegung
\item<3-> Joukowski-Auftriebsformel
\item<4-> Basler Problem
\item<5-> Produktentwicklungen
\item<6-> Zweidimensionale Elektrostatik
\item<7-> Gamma-Funktion $\Gamma(z)$
\item<8-> Venn-Konfigurationen und Inversion
\item<9-> Aircraft Flight Dynamics
\item<10-> Biharmonische Gleichung
\item<11-> Bessel-Differentialgleichung
\item<12-> Die Summe $\sum k$ und $\zeta(z)$
\end{enumerate}
\end{column}
}
\only<13-30>{
\begin{column}{0.48\textwidth}
\begin{enumerate}
\setcounter{enumi}{11}
\item<13-> Weyls Nullstellenalgorithmus
\item<14-> Fresnel-Integrale
\item<15-> Das dritte keplersche Gesetz
\item<16-> Laplace-Inversion und Bromwich Integral
\item<17-> Hauptwert eines divergenten Integrals
\item<18-> Gemeinsame Nullstellen und die Resultante
\item<19-> Julia-Mengen und Mandelbrot-Menge
\item<20-> Möbius-Transformation der komplexen Ebene
\item<21-> Riemann-$\zeta$-Funktion und Hankel-Contour
\end{enumerate}
\end{column}
}
\only<22->{%
\begin{column}{0.48\textwidth}
\begin{enumerate}
\setcounter{enumi}{21}
\item<22-> Der riemannsche Abbildungssatz
\item<23-> Elliptische Integrale
\item<24-> Stirling-Formel
\item<25-> Elliptische Funktionen als doppelt periodische Funktionen
\item<26-> Komplexe Zahlen und geometrische Konsruktionen in der Ebene
\item<27-> Padé-Approximation
\item<28-> Poisson-Formel und Harnack-Ungleichung
\item<29-> Weierstrass-Theorie
%\item<30-> 
\end{enumerate}
\end{column}
}
%\only<29->{%
%\begin{column}{0.48\textwidth}
%\begin{enumerate}
%\setcounter{enumi}{27}
%\end{enumerate}
%\end{column}
%}
\end{columns}
\end{frame}
\egroup
