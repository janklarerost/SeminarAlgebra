%
% vorwort.tex -- Vorwort zum Buch zum Seminar
%
% (c) 2019 Prof Dr Andreas Mueller, Hochschule Rapperswil
%
\chapter*{Vorwort}

\noindent
Dieses Buch entstand im Rahmen des Mathematischen Seminars
im Frühjahrssemester 2026 an der Ostschweizer Fachhochschule in Rapperswil.
Die Teilnehmer, Studierende der Studiengänge für Elektrotechnik, Informatik,
Erneuerbare Energien und Umwelttechnik und Bauingenieurwesen
der OST, erarbeiteten nach einer Einführung in das Themengebiet jeweils
einzelne Aspekte des Gebietes in Form einer Seminararbeit, über
deren Resultate sie auch in einem Vortrag informierten. 

Im Frühjahr 2026 waren die vielfältigen Synergien zwischen Algebra
und Analysis das Thema des Seminars.
In der komplexen Analysis zeigt die Tatsache, dass sich alle
differenzierbaren Funktionen immer beliebig genau durch Polynome
approximieren lassen, zum ersten Mal, dass alle Funktionen allein
durch Grenzübergang von rein algebraischen Funktionen erreicht werden
kann.
Eine solche Aussage trifft für die reelle Analysis sicher nicht zu.
Die Entdeckungen von Joseph Liouville im 19.~Jahrhundert zeigten dann,
dass sich diese Nähe zur Algebra nicht nur die Basis für alle praktischen
Berechnungen in der Anaysis ist, sondern sogar Aussagen darüber ermöglicht,
ob ein Integral überhaupt in geschlossener Form ausführbar ist.
Im 20.~Jahrhundert wurde mit dem Aufkommen von Computeralgebrasystemen
der Ruf nach Algorithmen laut, die die Automatisierung solcher
Entscheidungen ermöglichen.

Der erste Teil des Buches erarbeitet daher zunächst die Grundlagen
der komplexen Analysis.
Es wird gezeigt, dass komplex differenzierbare Funktionen immer
analytisch sind, also eine konvergente Potenzreihenentwicklung haben.
Dann wird gezeigt, wie die algebraischen Algorithmen zur Integration,
wie zum Beispiel der Bernoulli-Algorithmus, der die Integration
gebrochen rationaler Funktionen erlaubt und den man
schon im zweiten Semester lernt, sich verfeinern lassen.
Es wird definiert, was eine elementare Funktion ist und wie
der Risch-Algorithmus entscheiden kann, ob eine Funktion eine
Stammfunktion in geschlossener Form hat.

Im zweiten Teil erarbeiten die Teilnehmer des Seminars in individuellen
Arbeiten ein breites Spektrum von weiterführenden Fragestellungen
und Anwendungen.
In einigen Arbeiten wurde auch Code zur Demonstration der 
besprochenen Methoden und Resultate geschrieben, soweit
möglich und sinnvoll wurde dieser Code im Github-Repository
\index{Github-Repository}%
dieses Kurses%
\footnote{\url{https://github.com/AndreasFMueller/SeminarFelder.git}}
\cite{buch:repo}
abgelegt.
Im genannten Repository findet sich auch der Source-Code dieses
Skriptes, es wird hier unter einer Creative Commons Lizenz
zur Verfügung gestellt.

